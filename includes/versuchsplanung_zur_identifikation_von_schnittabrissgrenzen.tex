\chapter{Versuchsplanung zur Identifikation von Schnittabrissgrenzen für Ededlstahlbleche}
Ziel der Versuchsplanung ist die systematische Abgrenzung des prozesssicheren Arbeitsbereichs (Prozessfensters) beim Laserschneiden von Edelstahlblechen sowie die präzise Identifikation der Parameterbereiche, in denen Schnittabrisse auftreten. Unter einem Schnittabriss wird im Folgenden ein Zustand verstanden, in dem das Werkstück infolge ungeeigneter Parameterkombinationen nicht vollständig getrennt wird, weil der Schnittspalt lokal verschweißt oder die Schmelzaustragung unzureichend ist. Die so gewonnenen Grenzwerte bilden die Grundlage für die spätere Qualitätsmodellierung und dienen der Ableitung belastbarer Empfehlungen für robuste Parametersätze. Zugleich dient die hier beschriebene Vorgehensweise der Ableitung strukturierter Versuchspläne („Experimentalplans“) für die zentrale Versuchsdatenbank.

Die Experimente sind als zweidimensionale Rasterstudien ausgelegt, bei denen jeweils zwei der drei wesentlichen Prozessparameter – Laserleistung, Schnittgeschwindigkeit und Assistgasdruck – variiert werden, während der dritte Parameter konstant gehalten wird. Für jede untersuchte Parameterpaarung wird ein $3\times 3$-Feld gefertigt, in dem eine Größe entlang der horizontalen und die andere entlang der vertikalen Richtung schrittweise verändert wird. Formal seien die Stufen der beiden variierten Parameter $x\in\{x_1,x_2,x_3\}$ und $y\in\{y_1,y_2,y_3\}$; der jeweils dritte Parameter $z$ wird auf einem wohldefinierten Referenzniveau $z_0$ fixiert. Die Studie wird sequenziell für alle drei Kombinationen wiederholt (Leistung–Geschwindigkeit bei konstantem Gasdruck, Leistung–Gasdruck bei konstanter Geschwindigkeit sowie Geschwindigkeit–Gasdruck bei konstanter Leistung), sodass das Prozessfenster in den relevanten Teilräumen konsistent erfasst wird.

Die Wahl der Stufen $x_i$ und $y_j$ erfolgt material- und dickenspezifisch. Für jede untersuchte Blechdicke wird zunächst ein plausibler Arbeitsbereich aus praxisüblichen Startwerten und internen Erfahrungswerten abgeleitet und anschließend durch kurze Vorversuche verifiziert. Um Grenzbereiche mit ausreichender Auflösung zu erfassen, wird bei Bedarf adaptiv verfeinert (z.\,B. durch Intervallhalbierung zwischen zwei Stufen). Die Reihenfolge der Schnitte innerhalb eines Feldes wird randomisiert, um systematische Einflüsse durch fortschreitende Erwärmung, Düsenverschleiß oder driftende Maschinenzustände zu minimieren; die zentrale Kombination $(x_2,y_2)$ wird zusätzlich repliziert, um Prozess- und Messstreuungen abschätzen zu können.

Die praktische Durchführung jeder $3\times 3$-Matrix beginnt mit einer Funktionskontrolle der Anlage (Fokuslage, Düsenabstand, Strahlqualität, Gaszufuhr). Für jede Parameterkombination wird eine definierte Probengeometrie geschnitten. Ein Schnitt gilt als erfolgreich, wenn der Trennschnitt vollständig ist, wenn weder Durchhang noch Wiederaufschmelzen im Schnittspalt beobachtet wird und der Schmelzaustrag kontinuierlich erfolgt. Ein Schnittabriss liegt vor, wenn die Kontur nicht vollständig getrennt ist, der Schnittspalt lokal verschweißt oder ein charakteristischer Abbruch des Materialaustrags auftritt. Die Beurteilung erfolgt unmittelbar an der Maschine sowie nachgelagert in der Messzelle durch optische Inspektion und Dokumentation der Schnittkante. Die Kriterien werden vorab festgelegt und über alle Versuche konsistent angewendet, um Vergleichbarkeit sicherzustellen.

Sobald ein erster Grenzbereich identifiziert ist, wird das umliegende Parametergebiet gezielt erkundet. Hierzu werden die Stufen in Richtung des beobachteten Grenzverlaufs schrittweise angepasst, bis ein stabiler Übergang zwischen den Zuständen „Schnitt möglich“ und „Schnittabriss“ reproduzierbar nachgewiesen ist. Die so gewonnenen Grenzpunkte werden im jeweiligen Parameterraum verortet und bilden eine empirische Approximation des Prozessfensters. Für jede Kombination aus Blechdicke und Werkstoffgüte entsteht so eine Kontur, die den Bereich prozesssicherer Parameterkombinationen von instabilen Trennbedingungen trennt. Auf Basis dieser Voruntersuchungen werden anschließend detaillierte Versuchspläne („Experimentalplans“) ausgearbeitet. Diese beinhalten systematische Schneidversuche für Edelstahlbleche mit Dicken bis zu 20\,mm; dabei werden bewusst sowohl qualitativ hochwertige als auch minderwertige Schneidergebnisse erzeugt. Ziel ist der Aufbau einer umfassenden, kuratierten Datengrundlage für die Weiterentwicklung des KI-Modells.

Die beschriebene Vorgehensweise verfolgt zwei Ziele: Erstens erlaubt sie die belastbare Identifikation von Schnittabrissgrenzen und damit die sichere Festlegung eines prozessstabilen Arbeitsbereichs für Edelstahl. Zweitens liefert sie eine qualitativ wie quantitativ aussagekräftige Datenbasis, um das Zusammenspiel der Laserparameter zu verstehen und die Qualität von Schnittkante und Schnittfläche prädiktiv zu bewerten. Damit entsteht ein experimentell abgesichertes Fundament, das sowohl die spätere KI-Modellierung als auch eine reproduzierbare, datengetriebene Parametrierung in der Praxis unterstützt.
