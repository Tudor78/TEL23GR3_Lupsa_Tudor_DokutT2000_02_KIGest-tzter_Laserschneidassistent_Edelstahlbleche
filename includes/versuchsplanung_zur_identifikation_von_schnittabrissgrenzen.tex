\chapter{Identifikation der Schnittabrissgrenze und Datengenerierung}

Ziel dieses Kapitels ist die systematische Abgrenzung des prozesssicheren Arbeitsbereichs beim Laserschneiden von Edelstahlblechen sowie die präzise Identifikation der Parameterbereiche, in denen Schnittabrisse auftreten. Unter einem Schnittabriss wird im Folgenden ein Zustand verstanden, in dem das Werkstück infolge ungeeigneter Parameterkombinationen nicht vollständig getrennt wird, weil der Schnittspalt lokal verschweißt oder die Schmelzaustragung unzureichend ist. Die auf diese Weise gewonnenen Grenzwerte bilden die Grundlage für belastbare Parametrierungsempfehlungen und fließen zugleich in die Ausarbeitung strukturierter Versuchspläne zur Datengenerierung ein.

Die Experimente sind als zweidimensionale Rasterstudien ausgelegt, bei denen jeweils zwei der drei wesentlichen Prozessparameter, in dem Fall der Arbeit ist es die Fokuslage, die Schnittgeschwindigkeit und dem Gasdruck, variiert werden, während der dritte Parameter konstant gehalten wird. Die Laserleistung bleibt in diesen Studien konstant. Für jede untersuchte Parameterpaarung wird ein $3\times 3$-Feld gefertigt, in dem eine Größe entlang der horizontalen und die andere entlang der vertikalen Richtung stufenweise verändert wird. Formal seien die Stufen der beiden variierten Parameter $x\in\{x_1,x_2,x_3\}$ und $y\in\{y_1,y_2,y_3\}$. Der jeweils dritte Parameter $z$ ist auf einem Referenzniveau $z_0$ fixiert. Die Studie wird sequenziell für alle drei Kombinationen wiederholt, sodass das Prozessfenster in den relevanten Teilräumen konsistent erfasst wird. Die Wahl der Stufen erfolgt material- und dickenspezifisch, aus praxisüblichen Startwerten und internen Erfahrungswerten wird ein plausibler Arbeitsbereich abgeleitet.

Für jede Parameterkombination wird ein Viereck geschnitten. Ein Schnitt gilt als erfolgreich, wenn der Trennschnitt vollständig ist, weder Durchhang noch Wiederaufschmelzen im Schnittspalt beobachtet wird und der Schmelzaustrag kontinuierlich erfolgt. Die Beurteilung erfolgt unmittelbar an der Maschine sowie nachgelagert in der Messzelle durch optische Inspektion und Dokumentation der Schnittkante. Sobald ein erster Grenzbereich identifiziert ist, wird das umliegende Parametergebiet gezielt erkundet, bis ein stabiler Übergang zwischen den Zuständen „Schnitt möglich“ und „Schnittabriss“ reproduzierbar nachgewiesen ist. Die so gewonnenen Grenzpunkte werden im jeweiligen Parameterraum verortet und bilden eine aus Messdaten abgeleitete Näherung des Prozessfensters je Blechdicke.

Auf Basis dieser Grenzanalysen erfolgt die Datengenerierung in Form strukturierter Experimentalpläne. Hierfür wurden insgesamt 17 Edelstahlblechtafeln (1.5\,m $\times$ 2\,m) in den Dicken 5\,mm, 8\,mm, 10\,mm, 15\,mm und 20\,mm eingesetzt. Auf jeder Tafel wurden 128 quadratische Proben (100\,mm $\times$ 100\,mm) geschnitten, sodass ein Gesamtdatenumfang von 2\,176 Bauteilen entstand. Die Schneidparameter wurden pro Bauteil innerhalb der vorab definierten, dickenspezifischen Grenzen variiert und nach einem vorgegebenen Schema zufällig ausgewählt. Dieses Design stellt sicher, dass das Datenset sowohl hochwertige als auch ausdrücklich minderwertige Schneidergebnisse enthält, einschließlich Fehlschnitten und Parameterkombinationen nahe der Schnittabrissgrenze. Solche Negativbeispiele sind für die spätere Modellierung essentiell, um die Trennschärfe zwischen »prozesssicher« und »instabil« zu erhöhen und Fehlklassifikationen zu vermeiden.

Fehlschnitte wurden vollständig protokolliert. Auch wenn betroffene Proben in Einzelfällen nicht aus der Großtafel entnommen werden konnten, gingen diese Versuche mit eindeutiger Kennzeichnung in die Datenbank ein; damit ist bekannt, dass die jeweilige Parameterkombination für die gegebene Blechdicke kein akzeptables Schneidergebnis liefert. Sämtliche entnehmbaren Bauteile werden in der Messzelle vermessen, identifiziert und mit ihren Soll-/Ist-Parametern verknüpft. Die Datenmenge und der Versuchsplan wurden auf Basis der Erfahrungen mit dem bereits für Baustahl trainierten Modell gewählt, um eine ausreichende Abdeckung des Parameterraums und eine robuste Generalisierungsfähigkeit für Edelstahl sicherzustellen.

In Summe ermöglicht die kombinierte Vorgehensweise aus Grenzbestimmung und gezielter Datengenerierung sowohl die belastbare Identifikation der Schnittabrissgrenzen als auch den Aufbau einer ausgewogenen Datenbasis. Diese bildet die Voraussetzung für die Erweiterung und Validierung des KI-Modells, das künftig die Qualität von Edelstahlschnitten prädiktiv bewerten und prozesssichere Parameterbereiche verlässlich empfehlen soll.
