\chapter{Einführung}
\section{Zielsetzung}
Das Ziel dieser Projektarbeit ist es, ein bestehendes KI-gestütztes Assistenzsystem zu erweitern, das aktuell die Qualität beim Laserschneiden von Baustahlblechen vorhersagt und optimiert. Konkret soll die Leistungsfähigkeit dieses Modells auf Edelstahlbleche übertragen werden. Das aktuelle KI-Modell weist in Bezug auf Edelstahl Defizite auf, da es bisher nur mit Datensätzen von Baustahl trainiert wurde und die spezifischen Eigenschaften von Edelstahl unzureichend berücksichtigt. Die Defizite sind insbesondere in der Vorhersagegenauigkeit der Schnittqualität bei Edelstahlblechen erkennbar.
Um diese Lücke zu schließen, sollen neue, speziell auf Edelstahl zugeschnittene Trainingsdaten generiert und in das bestehende Modell integriert werden. Diese Daten erfassen insbesondere typische Eigenschaften wie Schneidgratbildung und Oberflächenrauheit. Zusätzlich werden bestehende optische Messmethoden geprüft und gegebenenfalls angepasst, um ihre Eignung für Edelstahl sicherzustellen. Demnach können die neu generierten Daten vermessen und in die Datenbank für das KI-Modell eingepflegt werden. Das Ziel ist ein robustes und zuverlässiges KI-Modell, das die Qualität von Edelstahlschnitten ebenso präzise vorhersagen kann wie bereits für Baustahl.

\section{Vorgehensweise}
Zunächst wird ein systematischer Testplan erstellt, um wichtige Schneidparameter, insbesondere Laserleistung, Schnittgeschwindigkeit, Gasdruck und Fokuslage, für verschiedene Edelstahldicken zu untersuchen. Ziel dieses Testplans ist es, Parameterbereiche zu identifizieren, die zu sogenannten Schnittabrissen führen. Ein Schnittabriss entsteht, wenn durch ungünstige Schneidparameter das Werkstück nicht vollständig getrennt wird, da die Schnittkante verschweißt.

Nachdem kritische Parameterbereiche, welche zu schlechten Schneidergebnissen führen, ermittelt wurden, werden detaillierte Versuchspläne („Experimentalplans“) entwickelt für die Datengenerierung. Diese umfassen systematische Schneidversuche für Edelstahlbleche mit Dicken bis zu 20 mm. Hierbei werden bewusst sowohl qualitativ hochwertige als auch minderwertige Schneidergebnisse erzeugt. Dies dient dazu, eine umfassende Datengrundlage für die Weiterentwicklung des KI-Modells bereitzustellen.

Die generierten Schneidproben werden anschließend in einer Messzelle vermessen, um die resultierenden Schnittkanten in die KI-Datenbank aufzunehmen. Da die Messmethoden ursprünglich für Baustahl entwickelt wurden, müssen sie für Edelstahl angepasst werden. Dies betrifft insbesondere die Kalibrierung und Einrichtung des Handscanner-Setups, welches Bilder der Schnittkanten für die Qualitätsschätzung aufnimmt. Ebenso muss die Vektorberechnung des eingesetzten 3D-Punktwolkenscanners optimiert werden. Da Edelstahlschnittkanten typischerweise ausgeprägtere Schneidgrate aufweisen als die von Baustahlblechen, muss die Umrechnung der 3D-Punktwolke in einen 2D-Vektor entsprechend angepasst werden, um die tatsächlichen Merkmale der Schnittkanten präzise abzubilden.
Neben der quantitativen Messung des Schneidgrats wird auch die Oberflächenrauheit qualitativ bewertet.
Die aufbereiteten Messdaten fließen anschließend in die Erweiterung und das Training des bestehenden KI-Modells ein. Ziel ist, dass dieses Modell anschließend die Qualität der Laserschneidkanten bei Edelstahlblechen zuverlässig vorhersagen kann. Nach erfolgreicher Implementierung erfolgen Validierungstests sowie weitere gezielte Optimierungen, um die Vorhersagequalität kontinuierlich zu verbessern und sicherzustellen, dass das gewählte Parameterset bereits vor dem Schneidprozess zuverlässig bewertet werden kann.