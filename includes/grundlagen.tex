\chapter{Grundlagen}
\section{Laserschneiden von Edelstahl}
Das Laserschneiden von Edelstahlblechen ist ein hochpräzises thermisches Trennverfahren, bei dem ein stark gebündelter Laserstrahl die Materialoberfläche lokal erwärmt und zum Schmelzen bringt. Im Anschluss wird die geschmolzene Zone durch ein Assistenzgas aus dem Schnittspalt gedrückt, wodurch eine präzise Kante erzielt wird. Im Vergleich zu Baustahl erfordert Edelstahl aufgrund seiner spezifischen Legierungsbestandteile, die zu einer geringeren Wärmeleitfähigkeit und höheren Reflexionsrate führen, eine Anpassung der Prozessparameter. Um den Energieverlust durch Reflexion auszugleichen, muss die Laserausgangsleistung erhöht werden, während gleichzeitig die Schnittgeschwindigkeit präzise kalibriert wird, um eine gleichmäßige Schmelz- und Materialabtragsrate sicherzustellen.
Ein weiterer entscheidender Einflussfaktor ist die korrekte Einstellung der Fokustiefe, da eine Abweichung sowohl zu unregelmäßigen Schnittkanten als auch zu erhöhter Gratbildung führen kann. Die Wahl des Assistenzgases und dessen Druck sind wesentliche Faktoren, die es zu berücksichtigen gilt. Stickstoff verhindert Oxidationsprozesse an der Schnittkante und trägt zu einer glatteren Oberfläche bei, während Sauerstoff zwar höhere Schnittgeschwindigkeiten ermöglicht, jedoch oft mit einer dunklen, oxidierten Kante einhergeht. Die Erreichung angemessener Gasdruckwerte ist essenziell, um den effektiven Abtransport von Schlacke zu gewährleisten, ohne dabei die Schnittkante mechanisch zu beeinträchtigen.
Darüber hinaus sind bei Edelstahlblechen materialbedingte Besonderheiten zu berücksichtigen. Die Tendenz zur fokussierten Applikation von Hitze resultiert oft in ausgeprägteren Schneidgraten, was die Nachbearbeitung erschwert und somit zu einem erhöhten Aufwand führt. Darüber hinaus kann die hohe Reflexionsrate bei gängigen Wellenlängen zu einer Reduktion der effektiven Energiedichte führen. In solchen Fällen können gepulste Laserquellen oder spezielle Strahlführungssysteme zum Einsatz kommen. Vor diesem Hintergrund müssen alle relevanten Parameter im Zusammenspiel abgestimmt werden, um sowohl die Schnittqualität als auch die Wirtschaftlichkeit des Verfahrens zu optimieren.

\begin{figure}[htbp]
    \centering
    \includegraphics[width=0.7\textwidth]{DHBW_logo.jpg}
    \caption{HMI-Oberfläche des Laserschneidprozesses}
    \label{fig:DHBW_logo}
\end{figure}
In Abbildung \ref{fig:DHBW_logo} ist die HMI-Oberfläche des Laserschneidprozesses abgebildet. Sie stellt die wesentlichen Prozessparameter – wie Laserleistung, Schnittgeschwindigkeit und Assistgasdruck – übersichtlich dar und bildet damit die Grundlage für die im Rahmen dieser Arbeit durchgeführten Testpläne. Anhand dieser Testpläne werden die gezeigten Einstellungen systematisch variiert und ihre Auswirkung auf die Schnittqualität von Edelstahlblechen untersucht.

\section{Grundlagen der Künstlichen Intelligenz}


\section{Cutting Edge AI}


\section{Messzelle und optische Messmethoden}
