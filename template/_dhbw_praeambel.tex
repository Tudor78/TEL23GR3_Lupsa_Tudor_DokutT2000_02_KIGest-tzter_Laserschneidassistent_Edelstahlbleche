%%% Präambel %%%
% Sprachumschaltung DE/EN
\usepackage{ifthen}
\newcommand{\DEoEN}[2]{\ifthenelse{\equal{\meineSprache}{DE}}{#1}{#2}}

% Zeichencodierung/Fonts
\usepackage[utf8]{inputenc}
\usepackage[T1]{fontenc}

% Farben + Code-Listings
\usepackage{xcolor}      % statt 'color' (mehr Features)
\usepackage{listings}
% Grund-Setup für Listings (Umlaute/EUR etc.)
\lstset{numbers=left, numberstyle=\tiny, numbersep=5pt, texcl=true}
\lstset{literate=
{Ö}{{\"O}}1 {Ä}{{\"A}}1 {Ü}{{\"U}}1 {ß}{{\ss}}2
{ü}{{\"u}}1 {ä}{{\"a}}1 {ö}{{\"o}}1 {€}{{\euro}}1
}
% Lesbares Listing-Layout + automatischer Zeilenumbruch
\lstdefinestyle{code}{
  numbers=left, numberstyle=\tiny, numbersep=5pt,
  breaklines=true, breakatwhitespace=true,
  columns=fullflexible, keepspaces=true, tabsize=2,
  basicstyle=\ttfamily\small,
  postbreak=\mbox{\textcolor{gray}{$\hookrightarrow$}}\space
}
\lstset{style=code}

% Seitenränder
\usepackage[
  left=2.5cm, right=2.5cm,
  top=2.5cm, bottom=2.5cm,
  foot=12mm, includefoot
]{geometry}

% Sprache + Anführungszeichen
\DEoEN{
  \usepackage[ngerman]{babel}
  \usepackage[babel,german=quotes]{csquotes}
}{
  \usepackage[english]{babel}
  \usepackage[babel,english=british]{csquotes}
}

% Listen, Grafiken, Zeilenabstand
\usepackage{enumerate}
\usepackage{graphicx}
\graphicspath{{img/}}
\usepackage[onehalfspacing]{setspace}

% Testtext, Akronyme
\usepackage{blindtext}
% \usepackage{color}  % durch xcolor ersetzt
\usepackage[nohyperlinks]{acronym}

% Literatur (biblatex + DHBW-Config)
\usepackage[
  backend=biber,
  bibstyle=_dhbw_authoryear,maxbibnames=99,
  citestyle=authoryear, dashed=false,
  uniquename=true, useprefix=true,
  bibencoding=utf8
]{biblatex}
%kein Punkt am Ende bei \footcite
%http://www.golatex.de/footcite-ohne-punkt-am-schluss-t4865.html
\renewcommand{\bibfootnotewrapper}[1]{\bibsentence#1}

% Bibliographie: Vornamen ausgeschrieben
\DeclareNameAlias{author}{family-given}
\DeclareNameAlias{editor}{family-given}

%Reihenfolge der Autorennamen
%   
% http://golatex.de/viewtopic,p,80448.html#80448
% Argumente: siehe http://texwelt.de/blog/modifizieren-eines-biblatex-stils/
\DeclareNameFormat{sortname}{% Bibliographie
  \ifnum\value{uniquename}=0 % Normalfall
    \ifuseprefix%
      {%
         \usebibmacro{name:family-given}
           {\namepartfamily}
           {\namepartgiveni}
           {\namepartprefix}
           {\namepartsuffixi}%
       }
      {%
         \usebibmacro{name:family-given}
           {\namepartfamily}
           {\namepartgiveni}
           {\namepartprefixi}
           {\namepartsuffixi}%
       }%
  \fi
  \ifnum\value{uniquename}=1% falls nicht eindeutig, abgek. Vorname 
      {%
         \usebibmacro{name:family-given}
           {\namepartfamily}
           {\namepartgiveni}
           {\namepartprefix}
           {\namepartsuffix}%
       }%
  \fi
  \ifnum\value{uniquename}=2% falls nicht eindeutig, ganzer Vorname 
      {%
         \usebibmacro{name:family-given}
           {\namepartfamily}
           {\namepartgiven}
           {\namepartprefix}
           {\namepartsuffix}%
       }%
  \fi   
  \usebibmacro{name:andothers}}

\DeclareNameFormat{labelname}{% für Zitate
  \ifnum\value{uniquename}=0 % Normalfall
    \ifuseprefix%
      {%
         \usebibmacro{name:family-given}
           {\namepartfamily}
           {\empty}
           {\namepartprefix}
           {\namepartsuffixi}%
       }
      {%
         \usebibmacro{name:family-given}
           {\namepartfamily}
           {\empty}
           {\namepartprefixi}
           {\namepartsuffixi}%
       }%
  \fi
  \ifnum\value{uniquename}=1% falls nicht eindeutig, abgek. Vorname 
      {%
         \usebibmacro{name:family-given}
           {\namepartfamily}
           {\namepartgiveni}
           {\namepartprefix}
           {\namepartsuffix}%
       }%
  \fi
  \ifnum\value{uniquename}=2% falls nicht eindeutig, ganzer Vorname 
      {%
         \usebibmacro{name:family-given}
           {\namepartfamily}
           {\namepartgiven}
           {\namepartprefix}
           {\namepartsuffix}%
       }%
  \fi   
  \usebibmacro{name:andothers}}
      
  
\DeclareFieldFormat{extrayear}{% = the 'a' in 'Jones 1995a'
  \iffieldnums{labelyear}
    {\mknumalph{#1}}
    {\mknumalph{#1}}}        

% Namen getrennt durch Komma (Zitate)
\DeclareDelimFormat*[footcite,cite,textcite,parencite]{multinamedelim}{\addcomma\space}
% bzw. Semikolon (Literaturverzeichnis)
\DeclareDelimFormat[bib,biblist]{multinamedelim}{\addsemicolon\space}
% keine besondere Behandlung beim letzten Autor
\DeclareDelimAlias{finalnamedelim}{multinamedelim}
%\DeclareDelimAlias{multilistdelim}{multinamedelim}

\renewcommand{\nameyeardelim}{~}

% Literaturverzeichnis: Doppelpunkt zwischen Name (Jahr): Rest 
% http://de.comp.text.tex.narkive.com/Tn1HUIXB/biblatex-authoryear-und-doppelpunkt
\renewcommand{\labelnamepunct}{\addcolon\addspace}

% damit die Darstellung für Vollzitate von Primärquellen in 
% Fußnoten später auf "nicht fett" geändert werden kann 
% (nur für Zitate von Sekundärliteratur relevant)
\newcommand{\textfett}[1]{\textbf{#1}}

% für Zitate von Sekundärliteratur:
\newcommand{\footcitePrimaerSekundaer}[4]{%
  \renewcommand{\textfett}[1]{##1}%
  \footnote{\fullcite[#2]{#1}, \DEoEN{zitiert nach}{as cited in} \cite[#4]{#3}}%  
  \renewcommand{\textfett}[1]{\textbf{##1}}%
}

% Im Literaturverzeichnis: Autor (Jahr) fett
\renewbibmacro*{author}{%
  \ifboolexpr{%
    test \ifuseauthor%
    and
    not test {\ifnameundef{author}}
  }
    {\usebibmacro{bbx:dashcheck}
       {\bibnamedash}
       {\usebibmacro{bbx:savehash}%
        \textfett{\printnames{author}}%
        \iffieldundef{authortype}
          {\setunit{\addspace}}
          {\setunit{\addcomma\space}}}%
     \iffieldundef{authortype}
       {}
       {\usebibmacro{authorstrg}%
        \setunit{\addspace}}}%
    {\global\undef\bbx@lasthash
     \usebibmacro{labeltitle}%
     \setunit*{\addspace}}%
  \textfett{\usebibmacro{date+extrayear}}}

% Sonderfall: Quelle ohne Autor, aber mit Herausgeber
% Name des Herausgebers wird fett gedruckt
\renewbibmacro*{bbx:editor}[1]{%
  \ifboolexpr{%
    test \ifuseeditor%
    and
    not test {\ifnameundef{editor}}
  }
    {\usebibmacro{bbx:dashcheck}
       {\bibnamedash}
       {\textfett{\printnames{editor}}%
        \setunit{\addcomma\space}%
        \usebibmacro{bbx:savehash}}%
     \usebibmacro{#1}%
     \clearname{editor}%
     \setunit{\addspace}}%
    {\global\undef\bbx@lasthash
     \usebibmacro{labeltitle}%
     \setunit*{\addspace}}%
  \textfett{\usebibmacro{date+extrayear}}}

\DefineBibliographyStrings{ngerman}{% Anpassungen für deutsche Sprache
	nodate = {{o.J.}},
	urlseen = {{Abruf:}},
	ibidem = {{ebenda}},
	andothers = {{et\addabbrvspace al\adddot}}
}
\DefineBibliographyStrings{english}{% Anpassungen für englische Sprache
    nodate = {{w.y.}},
    urlseen = {{retrieval:}}
}

% keine Anführungszeichen beim Titel im Literaturverzeichnis
\DeclareFieldFormat[article,book,inbook,inproceedings,manual,misc,phdthesis,thesis,online,report]{title}{#1\isdot}

\newcommand{\literaturverzeichnis}{%
% nur Literaturverzeichnis
% (als eigenes Kapitel)
\phantomsection
\addcontentsline{toc}{chapter}{\refname}
\spezialkopfzeile{\refname}
\defbibheading{lit}{\chapter*{\refname}}
\label{chapter:quellen}
\printbibliography[heading=lit,notkeyword=ausblenden]
}


% Verzeichnisse/Tabellen
\usepackage{tocloft}
\usepackage{multirow}
\usepackage{amsmath}
\usepackage{amssymb}
\usepackage{booktabs}

% Hyperlinks
\usepackage[hypertexnames=false]{hyperref}

% Bessere Umbrüche in \url{…}
\usepackage{xurl}
\urlstyle{tt}

% Intelligente Referenzen (nach hyperref laden!)
\usepackage[capitalise,nameinlink,noabbrev]{cleveref}
% "Listing" -> "Quellcode" (auch für Verzeichnis und Verweise)
\renewcommand{\lstlistingname}{Quellcode}
\renewcommand{\lstlistlistingname}{Quellcodeverzeichnis}
\providecommand*{\lstlistingautorefname}{Quellcode}
\crefname{lstlisting}{Quellcode}{Quellcode}
\Crefname{lstlisting}{Quellcode}{Quellcode}

% Anhangszähler/Verzeichnis (wie in deiner Vorlage)
\newcounter{anhcnt}\setcounter{anhcnt}{0}
\newlistof{anhang}{app}{}
\newcommand{\anhang}[1]{%
  \refstepcounter{anhcnt}\setcounter{anhteilcnt}{0}
  \section*{\appendixname\ \theanhcnt: #1}
  \addcontentsline{app}{section}{\protect\numberline{\appendixname\ \theanhcnt}#1}\par
}
\newcounter{anhteilcnt}\setcounter{anhteilcnt}{0}
\newcommand{\anhangteil}[1]{%
  \refstepcounter{anhteilcnt}
  \subsection*{\appendixname\ \arabic{anhcnt}/\arabic{anhteilcnt}: #1}
  \addcontentsline{app}{subsection}{\protect\numberline{\appendixname\ \theanhcnt/\arabic{anhteilcnt}}#1}\par
}
\renewcommand{\theanhteilcnt}{\appendixname\ \theanhcnt/\arabic{anhteilcnt}}

% tocloft-Einrückungen für Anhangverzeichnis
\makeatletter
\newcommand{\abstaendeanhangverzeichnis}{%
  \renewcommand*{\l@section}{\@dottedtocline{1}{0em}{5.5em}}
  \renewcommand*{\l@subsection}{\@dottedtocline{2}{2.3em}{6.5em}}
}
% Einträge LOF/LOT und Quellcodeverzeichnis (LOL) angeglichen
\renewcommand*{\l@figure}{\@dottedtocline{1}{0em}{2.3em}}
\renewcommand*{\l@table}{\@dottedtocline{1}{0em}{2.3em}}
\renewcommand*{\l@lstlisting}{\@dottedtocline{1}{0em}{2.3em}}
\makeatother

% Fortlaufende Zähler über Kapitel hinweg
\usepackage{chngcntr}
\counterwithout{figure}{chapter}
\counterwithout{table}{chapter}
\counterwithout{footnote}{chapter}

% Kopfzeilen (KOMA-Script)
\usepackage[automark]{scrlayer-scrpage}
%% Definitionen für Kopf- und Fußzeile auf normalen Seiten
\defpagestyle{kopfzeile}
{% Kopfdefinition
  (\textwidth,0pt)    % Länge der oberen Linie,Dicke der oberen Linie       
  {} % Definition für linke Seiten im doppelseitigen Layout
  {} % Definition für rechte Seiten im doppelseitigen Layout      
  {  % Definition für Seiten im einseitigen Layout
	\makebox[0pt][l]{\rightmark}% 
	\makebox[\linewidth]{}% 
  }        
  (\textwidth, 0.4pt) % Untere Linienlänge, Untere Liniendicke
}
{% Fußdefinition
  (\textwidth,0pt)    % Obere Linienlänge, Obere Liniendicke
  {} % Definition für linke Seiten im doppelseitigen Layout
  {} % Definition für rechte Seiten im doppelseitigen Layout
  {  % Definition für Seiten im einseitigen Layout
    \makebox[\linewidth]{}%
    \makebox[0pt][r]{\pagemark}%
  }
  (\textwidth, 0pt)   % Länge der unteren Linie,Dicke der unteren Linie
}

%% Definitionen für Kopf- und Fußzeile auf ersten Seiten eines Kapitels
\defpagestyle{kapitelkopfzeile}
{% Kopfdefinition
  (\textwidth,0pt)    % Länge der oberen Linie,Dicke der oberen Linie       
  {} % Definition für linke Seiten im doppelseitigen Layout
  {} % Definition für rechte Seiten im doppelseitigen Layout      
  {}  % Definition für Seiten im einseitigen Layout
  (\textwidth, 0pt) % Untere Linienlänge, Untere Liniendicke
}
{% Fußdefinition
  (\textwidth,0pt)    % Obere Linienlänge, Obere Liniendicke
  {} % Definition für linke Seiten im doppelseitigen Layout
  {} % Definition für rechte Seiten im doppelseitigen Layout
  {  % Definition für Seiten im einseitigen Layout
    \makebox[\linewidth]{}%
    \makebox[0pt][r]{\pagemark}%
  }
  (\textwidth, 0pt)   % Länge der unteren Linie,Dicke der unteren Linie
}

%% Definitionen für Kopf- und Fußzeile im Anhang und bei Quellenverzeichnisse
\newcommand{\spezialkopfzeileBezeichnung}{}
\defpagestyle{spezialkopfzeile}
{% Kopfdefinition
  (\textwidth,0pt)    % Länge der oberen Linie,Dicke der oberen Linie       
  {} % Definition für linke Seiten im doppelseitigen Layout
  {} % Definition für rechte Seiten im doppelseitigen Layout      
  {  % Definition für Seiten im einseitigen Layout
	\makebox[0pt][l]{\spezialkopfzeileBezeichnung}% 
	\makebox[\linewidth]{}% 
  }        
  (\textwidth, 0.4pt) % Untere Linienlänge, Untere Liniendicke
}
{% Fußdefinition
  (\textwidth,0pt)    % Obere Linienlänge, Obere Liniendicke
  {} % Definition für linke Seiten im doppelseitigen Layout
  {} % Definition für rechte Seiten im doppelseitigen Layout
  {  % Definition für Seiten im einseitigen Layout
    \makebox[\linewidth]{}%
    \makebox[0pt][r]{\pagemark}%
  }
  (\textwidth, 0pt)   % Länge der unteren Linie,Dicke der unteren Linie
}
            
\newcommand\spezialkopfzeile[1]{%
  \renewcommand\spezialkopfzeileBezeichnung{#1}
  \pagestyle{spezialkopfzeile}
}
                
% Standard-Pagestyle auswählen
\pagestyle{kopfzeile}

% keine Kopfzeile anzeigen auf Seiten, auf denen ein 
% Kapitel beginnt oder das Inhalts-/Abbildungs-/Tabellenverzeichnis steht 
\renewcommand{\chapterpagestyle}{kapitelkopfzeile}
\tocloftpagestyle{kapitelkopfzeile}



% Euro-Zeichen
\usepackage{textcomp}
\usepackage{eurosym}
\renewcommand{\texteuro}{\euro}  % ACHTUNG: nach hyperref laden!

% Kompatibilität KOMA-Script
\usepackage{scrhack}

% Abstände bei Kapitelüberschriften (inkl. Verzeichnisse)
\renewcommand*\chapterheadstartvskip{\vspace*{-\topskip}}
\newcommand{\myBeforeTitleSkip}{1mm}
\newcommand{\myAfterTitleSkip}{10mm}
\setlength\cftbeforetoctitleskip{\myBeforeTitleSkip}
\setlength\cftbeforeloftitleskip{\myBeforeTitleSkip}
\setlength\cftbeforelottitleskip{\myBeforeTitleSkip}
\setlength\cftaftertoctitleskip{\myAfterTitleSkip}
\setlength\cftafterloftitleskip{\myAfterTitleSkip}
\setlength\cftafterlottitleskip{\myAfterTitleSkip}

% Anhang beginnen
\newcommand{\startAnhang}{%
  \chapter*{\appendixname}
  \addcontentsline{toc}{chapter}{\appendixname}
  \section*{\anhangVzBezeichnung}
  \vspace{-8em}
  % vor \listofanhang müssen Einrückungen angepasst werden
  \abstaendeanhangverzeichnis
  \spezialkopfzeile{\DEoEN{Anhang}{Appendix}}
}

% Abkürzungsverzeichnis beginnen
\newcommand{\startAbkVerzeichnis}{%
  \chapter*{\abkVzBezeichnung}
  \addcontentsline{toc}{chapter}{\abkVzBezeichnung}
}

% Zeilenabstand in Tabellen schnell ändern
\newcommand{\ra}[1]{\renewcommand{\arraystretch}{#1}}

% Sprachspezifische Überschriften
\DEoEN{%
  \newcommand{\abkVzBezeichnung}{Abkürzungsverzeichnis}
  \newcommand{\anhangVzBezeichnung}{Anhangverzeichnis}
  \renewcaptionname{ngerman}{\refname}{Literaturverzeichnis}
  \renewcaptionname{ngerman}{\figurename}{Abb.}
  \renewcaptionname{ngerman}{\tablename}{Tab.}
}{%
  \newcommand{\abkVzBezeichnung}{Abbreviations}
  \newcommand{\anhangVzBezeichnung}{Appendix directory}
  \renewcaptionname{english}{\contentsname}{Table of Contents}
  \renewcaptionname{english}{\figurename}{Fig.}
  \renewcaptionname{english}{\tablename}{Tab.}
}

%%% Ende der Präambel %%%
